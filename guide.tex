% Options for packages loaded elsewhere
\PassOptionsToPackage{unicode}{hyperref}
\PassOptionsToPackage{hyphens}{url}
\PassOptionsToPackage{dvipsnames,svgnames,x11names}{xcolor}
%
\documentclass[
  11pt,
  letterpaper,
  DIV=11,
  numbers=noendperiod]{scrartcl}

\usepackage{amsmath,amssymb}
\usepackage{iftex}
\ifPDFTeX
  \usepackage[T1]{fontenc}
  \usepackage[utf8]{inputenc}
  \usepackage{textcomp} % provide euro and other symbols
\else % if luatex or xetex
  \usepackage{unicode-math}
  \defaultfontfeatures{Scale=MatchLowercase}
  \defaultfontfeatures[\rmfamily]{Ligatures=TeX,Scale=1}
\fi
\usepackage{lmodern}
\ifPDFTeX\else  
    % xetex/luatex font selection
    \setmainfont[]{Times New Roman}
    \setsansfont[]{Arial}
    \setmonofont[]{Fira Mono}
\fi
% Use upquote if available, for straight quotes in verbatim environments
\IfFileExists{upquote.sty}{\usepackage{upquote}}{}
\IfFileExists{microtype.sty}{% use microtype if available
  \usepackage[]{microtype}
  \UseMicrotypeSet[protrusion]{basicmath} % disable protrusion for tt fonts
}{}
\makeatletter
\@ifundefined{KOMAClassName}{% if non-KOMA class
  \IfFileExists{parskip.sty}{%
    \usepackage{parskip}
  }{% else
    \setlength{\parindent}{0pt}
    \setlength{\parskip}{6pt plus 2pt minus 1pt}}
}{% if KOMA class
  \KOMAoptions{parskip=half}}
\makeatother
\usepackage{xcolor}
\setlength{\emergencystretch}{3em} % prevent overfull lines
\setcounter{secnumdepth}{5}
% Make \paragraph and \subparagraph free-standing
\makeatletter
\ifx\paragraph\undefined\else
  \let\oldparagraph\paragraph
  \renewcommand{\paragraph}{
    \@ifstar
      \xxxParagraphStar
      \xxxParagraphNoStar
  }
  \newcommand{\xxxParagraphStar}[1]{\oldparagraph*{#1}\mbox{}}
  \newcommand{\xxxParagraphNoStar}[1]{\oldparagraph{#1}\mbox{}}
\fi
\ifx\subparagraph\undefined\else
  \let\oldsubparagraph\subparagraph
  \renewcommand{\subparagraph}{
    \@ifstar
      \xxxSubParagraphStar
      \xxxSubParagraphNoStar
  }
  \newcommand{\xxxSubParagraphStar}[1]{\oldsubparagraph*{#1}\mbox{}}
  \newcommand{\xxxSubParagraphNoStar}[1]{\oldsubparagraph{#1}\mbox{}}
\fi
\makeatother

\usepackage{color}
\usepackage{fancyvrb}
\newcommand{\VerbBar}{|}
\newcommand{\VERB}{\Verb[commandchars=\\\{\}]}
\DefineVerbatimEnvironment{Highlighting}{Verbatim}{commandchars=\\\{\}}
% Add ',fontsize=\small' for more characters per line
\usepackage{framed}
\definecolor{shadecolor}{RGB}{241,243,245}
\newenvironment{Shaded}{\begin{snugshade}}{\end{snugshade}}
\newcommand{\AlertTok}[1]{\textcolor[rgb]{0.68,0.00,0.00}{#1}}
\newcommand{\AnnotationTok}[1]{\textcolor[rgb]{0.37,0.37,0.37}{#1}}
\newcommand{\AttributeTok}[1]{\textcolor[rgb]{0.40,0.45,0.13}{#1}}
\newcommand{\BaseNTok}[1]{\textcolor[rgb]{0.68,0.00,0.00}{#1}}
\newcommand{\BuiltInTok}[1]{\textcolor[rgb]{0.00,0.23,0.31}{#1}}
\newcommand{\CharTok}[1]{\textcolor[rgb]{0.13,0.47,0.30}{#1}}
\newcommand{\CommentTok}[1]{\textcolor[rgb]{0.37,0.37,0.37}{#1}}
\newcommand{\CommentVarTok}[1]{\textcolor[rgb]{0.37,0.37,0.37}{\textit{#1}}}
\newcommand{\ConstantTok}[1]{\textcolor[rgb]{0.56,0.35,0.01}{#1}}
\newcommand{\ControlFlowTok}[1]{\textcolor[rgb]{0.00,0.23,0.31}{\textbf{#1}}}
\newcommand{\DataTypeTok}[1]{\textcolor[rgb]{0.68,0.00,0.00}{#1}}
\newcommand{\DecValTok}[1]{\textcolor[rgb]{0.68,0.00,0.00}{#1}}
\newcommand{\DocumentationTok}[1]{\textcolor[rgb]{0.37,0.37,0.37}{\textit{#1}}}
\newcommand{\ErrorTok}[1]{\textcolor[rgb]{0.68,0.00,0.00}{#1}}
\newcommand{\ExtensionTok}[1]{\textcolor[rgb]{0.00,0.23,0.31}{#1}}
\newcommand{\FloatTok}[1]{\textcolor[rgb]{0.68,0.00,0.00}{#1}}
\newcommand{\FunctionTok}[1]{\textcolor[rgb]{0.28,0.35,0.67}{#1}}
\newcommand{\ImportTok}[1]{\textcolor[rgb]{0.00,0.46,0.62}{#1}}
\newcommand{\InformationTok}[1]{\textcolor[rgb]{0.37,0.37,0.37}{#1}}
\newcommand{\KeywordTok}[1]{\textcolor[rgb]{0.00,0.23,0.31}{\textbf{#1}}}
\newcommand{\NormalTok}[1]{\textcolor[rgb]{0.00,0.23,0.31}{#1}}
\newcommand{\OperatorTok}[1]{\textcolor[rgb]{0.37,0.37,0.37}{#1}}
\newcommand{\OtherTok}[1]{\textcolor[rgb]{0.00,0.23,0.31}{#1}}
\newcommand{\PreprocessorTok}[1]{\textcolor[rgb]{0.68,0.00,0.00}{#1}}
\newcommand{\RegionMarkerTok}[1]{\textcolor[rgb]{0.00,0.23,0.31}{#1}}
\newcommand{\SpecialCharTok}[1]{\textcolor[rgb]{0.37,0.37,0.37}{#1}}
\newcommand{\SpecialStringTok}[1]{\textcolor[rgb]{0.13,0.47,0.30}{#1}}
\newcommand{\StringTok}[1]{\textcolor[rgb]{0.13,0.47,0.30}{#1}}
\newcommand{\VariableTok}[1]{\textcolor[rgb]{0.07,0.07,0.07}{#1}}
\newcommand{\VerbatimStringTok}[1]{\textcolor[rgb]{0.13,0.47,0.30}{#1}}
\newcommand{\WarningTok}[1]{\textcolor[rgb]{0.37,0.37,0.37}{\textit{#1}}}

\providecommand{\tightlist}{%
  \setlength{\itemsep}{0pt}\setlength{\parskip}{0pt}}\usepackage{longtable,booktabs,array}
\usepackage{calc} % for calculating minipage widths
% Correct order of tables after \paragraph or \subparagraph
\usepackage{etoolbox}
\makeatletter
\patchcmd\longtable{\par}{\if@noskipsec\mbox{}\fi\par}{}{}
\makeatother
% Allow footnotes in longtable head/foot
\IfFileExists{footnotehyper.sty}{\usepackage{footnotehyper}}{\usepackage{footnote}}
\makesavenoteenv{longtable}
\usepackage{graphicx}
\makeatletter
\newsavebox\pandoc@box
\newcommand*\pandocbounded[1]{% scales image to fit in text height/width
  \sbox\pandoc@box{#1}%
  \Gscale@div\@tempa{\textheight}{\dimexpr\ht\pandoc@box+\dp\pandoc@box\relax}%
  \Gscale@div\@tempb{\linewidth}{\wd\pandoc@box}%
  \ifdim\@tempb\p@<\@tempa\p@\let\@tempa\@tempb\fi% select the smaller of both
  \ifdim\@tempa\p@<\p@\scalebox{\@tempa}{\usebox\pandoc@box}%
  \else\usebox{\pandoc@box}%
  \fi%
}
% Set default figure placement to htbp
\def\fps@figure{htbp}
\makeatother

\usepackage{tocloft}
\renewcommand{\cftsecfont}{\normalsize}
\renewcommand{\cftsecpagefont}{\normalsize}
\setlength{\cftbeforesecskip}{4pt}
\definecolor{codefontcolor}{RGB}{125,18,186}
\definecolor{codebggray}{HTML}{f8f9fa}
\let\textttOrig\texttt
\renewcommand{\texttt}[1]{\textttOrig{\colorbox{codebggray}{\textcolor{codefontcolor}{#1}}}}
\KOMAoption{captions}{tableheading}
\makeatletter
\@ifpackageloaded{tcolorbox}{}{\usepackage[skins,breakable]{tcolorbox}}
\@ifpackageloaded{fontawesome5}{}{\usepackage{fontawesome5}}
\definecolor{quarto-callout-color}{HTML}{909090}
\definecolor{quarto-callout-note-color}{HTML}{0758E5}
\definecolor{quarto-callout-important-color}{HTML}{CC1914}
\definecolor{quarto-callout-warning-color}{HTML}{EB9113}
\definecolor{quarto-callout-tip-color}{HTML}{00A047}
\definecolor{quarto-callout-caution-color}{HTML}{FC5300}
\definecolor{quarto-callout-color-frame}{HTML}{acacac}
\definecolor{quarto-callout-note-color-frame}{HTML}{4582ec}
\definecolor{quarto-callout-important-color-frame}{HTML}{d9534f}
\definecolor{quarto-callout-warning-color-frame}{HTML}{f0ad4e}
\definecolor{quarto-callout-tip-color-frame}{HTML}{02b875}
\definecolor{quarto-callout-caution-color-frame}{HTML}{fd7e14}
\makeatother
\makeatletter
\@ifpackageloaded{caption}{}{\usepackage{caption}}
\AtBeginDocument{%
\ifdefined\contentsname
  \renewcommand*\contentsname{Table of contents}
\else
  \newcommand\contentsname{Table of contents}
\fi
\ifdefined\listfigurename
  \renewcommand*\listfigurename{List of Figures}
\else
  \newcommand\listfigurename{List of Figures}
\fi
\ifdefined\listtablename
  \renewcommand*\listtablename{List of Tables}
\else
  \newcommand\listtablename{List of Tables}
\fi
\ifdefined\figurename
  \renewcommand*\figurename{Figure}
\else
  \newcommand\figurename{Figure}
\fi
\ifdefined\tablename
  \renewcommand*\tablename{Table}
\else
  \newcommand\tablename{Table}
\fi
}
\@ifpackageloaded{float}{}{\usepackage{float}}
\floatstyle{ruled}
\@ifundefined{c@chapter}{\newfloat{codelisting}{h}{lop}}{\newfloat{codelisting}{h}{lop}[chapter]}
\floatname{codelisting}{Listing}
\newcommand*\listoflistings{\listof{codelisting}{List of Listings}}
\makeatother
\makeatletter
\makeatother
\makeatletter
\@ifpackageloaded{caption}{}{\usepackage{caption}}
\@ifpackageloaded{subcaption}{}{\usepackage{subcaption}}
\makeatother

\usepackage{bookmark}

\IfFileExists{xurl.sty}{\usepackage{xurl}}{} % add URL line breaks if available
\urlstyle{same} % disable monospaced font for URLs
\hypersetup{
  pdftitle={Git and Github Guide},
  pdfauthor={Kaiyi Ye 35255005},
  colorlinks=true,
  linkcolor={blue},
  filecolor={Maroon},
  citecolor={Blue},
  urlcolor={Blue},
  pdfcreator={LaTeX via pandoc}}


\title{Git and Github Guide}
\author{Kaiyi Ye 35255005}
\date{}

\begin{document}
\maketitle

\renewcommand*\contentsname{Table of contents}
{
\hypersetup{linkcolor=}
\setcounter{tocdepth}{1}
\tableofcontents
}

\newpage

\section{Set Up RStudio Project and Create a Quarto
File}\label{set-up-rstudio-project-and-create-a-quarto-file}

\subsection{Step 1: Create a New
Project}\label{step-1-create-a-new-project}

\begin{enumerate}
\def\labelenumi{\arabic{enumi}.}
\tightlist
\item
  Open RStudio.
\item
  Go to the top menu and select \texttt{File} \textgreater{}
  \texttt{New\ Project....}
\item
  In the ``New Project'' wizard, select \texttt{New\ Directory}.
\item
  Choose \texttt{New\ Project}.
\item
  Enter a name for your project and select a location on your computer
  where you want to save it.
\item
  Click \texttt{Create\ Project}.
\end{enumerate}

\subsection{Step 2: Create a Quarto
File}\label{step-2-create-a-quarto-file}

\begin{enumerate}
\def\labelenumi{\arabic{enumi}.}
\tightlist
\item
  Go to the top menu and select \texttt{File} \textgreater{}
  \texttt{New\ file} \textgreater{} \texttt{Quarto\ Document}.
\item
  Save the file as \texttt{example.qmd}.
\item
  Modify the YAML to set the document to render as HTML:
\end{enumerate}

\begin{Shaded}
\begin{Highlighting}[]
\FunctionTok{title}\KeywordTok{:}\AttributeTok{ }\StringTok{"example"}
\FunctionTok{author}\KeywordTok{:}\AttributeTok{ }\StringTok{"your name"}
\FunctionTok{date}\KeywordTok{:}\AttributeTok{ today}
\FunctionTok{format}\KeywordTok{:}\AttributeTok{ html}
\end{Highlighting}
\end{Shaded}

\begin{enumerate}
\def\labelenumi{\arabic{enumi}.}
\setcounter{enumi}{3}
\tightlist
\item
  Save the file, then render to preview the document output (see
  Figure~\ref{fig-render-preview}).
\end{enumerate}

\begin{figure}

\centering{

\pandocbounded{\includegraphics[keepaspectratio]{Images/Screenshot-knitted-qmd.png}}

}

\caption{\label{fig-render-preview}Render Preview}

\end{figure}%

\newpage

\section{Create a Git Repository from Existing
work}\label{create-a-git-repository-from-existing-work}

If you have an existing project that you want to version control with
Git, follow these steps:

\subsection{Step 1: Initialize the Git
Repository}\label{step-1-initialize-the-git-repository}

\begin{enumerate}
\def\labelenumi{\arabic{enumi}.}
\tightlist
\item
  Open your terminal or command prompt
\item
  Navigate to the root directory of your existing project folder using
  the \texttt{cd} command. For example:
  \texttt{cd\ path/to/your/project}
\item
  Run \texttt{git\ init} to initialize this directory as a Git
  repository. This command creates a new subdirectory named
  \texttt{.git} that contains all of your necessary repository files.
\end{enumerate}

\subsection{Step 2: Add Your Files to the
Repository}\label{step-2-add-your-files-to-the-repository}

\begin{enumerate}
\def\labelenumi{\arabic{enumi}.}
\tightlist
\item
  Add all of your project files to the staging area:
  \texttt{git\ add\ .}
\item
  Commit the files to the repository with a descriptive message:
  \texttt{git\ commit\ -m\ "Initial\ commit"}
\end{enumerate}

\subsection{Step 3: Set Up the Remote
Repository}\label{step-3-set-up-the-remote-repository}

\begin{enumerate}
\def\labelenumi{\arabic{enumi}.}
\tightlist
\item
  Go to GitHub and create a new repository. Do not initialize it with a
  README, .gitignore, or license.
\item
  Copy the URL (SSH) of the new GitHub repository.
\item
  In your terminal, add the remote repository URL to your local
  repository:
\end{enumerate}

\texttt{git\ remote\ add\ origin\ git@github.com:your-username/your-repository.git}

\begin{enumerate}
\def\labelenumi{\arabic{enumi}.}
\setcounter{enumi}{3}
\tightlist
\item
  Push your local repository to GitHub:
  \texttt{git\ push\ -u\ origin\ main}
\end{enumerate}

Now you should be able to see your repository on GitHub.

\begin{tcolorbox}[enhanced jigsaw, opacityback=0, coltitle=black, bottomrule=.15mm, opacitybacktitle=0.6, breakable, colbacktitle=quarto-callout-tip-color!10!white, title=\textcolor{quarto-callout-tip-color}{\faLightbulb}\hspace{0.5em}{Tip}, left=2mm, rightrule=.15mm, toprule=.15mm, leftrule=.75mm, bottomtitle=1mm, toptitle=1mm, colback=white, titlerule=0mm, arc=.35mm, colframe=quarto-callout-tip-color-frame]

We use the \texttt{-u} flag to tell Git to remember the connection
between our local main branch and the remote origin/main branch.

\end{tcolorbox}

\newpage

\section{Create a New Branch and Make
Changes}\label{create-a-new-branch-and-make-changes}

\subsection{Step 1: Create a New
Branch}\label{step-1-create-a-new-branch}

\begin{enumerate}
\def\labelenumi{\arabic{enumi}.}
\tightlist
\item
  In the terminal, make sure you are inside your project folder:
  \texttt{cd\ my-new-project}
\item
  Create and switch to a new branch called \texttt{testbranch}:
\end{enumerate}

\begin{Shaded}
\begin{Highlighting}[]
\FunctionTok{git}\NormalTok{ branch tsetbranch}
\FunctionTok{git}\NormalTok{ switch testbranch}
\end{Highlighting}
\end{Shaded}

Or you can also do this in one command:

\begin{Shaded}
\begin{Highlighting}[]
\FunctionTok{git}\NormalTok{ switch }\AttributeTok{{-}c}\NormalTok{ testbranch}
\end{Highlighting}
\end{Shaded}

\begin{tcolorbox}[enhanced jigsaw, opacityback=0, coltitle=black, bottomrule=.15mm, opacitybacktitle=0.6, breakable, colbacktitle=quarto-callout-tip-color!10!white, title=\textcolor{quarto-callout-tip-color}{\faLightbulb}\hspace{0.5em}{Tip}, left=2mm, rightrule=.15mm, toprule=.15mm, leftrule=.75mm, bottomtitle=1mm, toptitle=1mm, colback=white, titlerule=0mm, arc=.35mm, colframe=quarto-callout-tip-color-frame]

The \texttt{-c} flag is short for \texttt{-\/-create}. It tells git to
create a new branch and immediately switch to it.

\end{tcolorbox}

\subsection{Step 2: Make Changes to a
File}\label{step-2-make-changes-to-a-file}

\begin{enumerate}
\def\labelenumi{\arabic{enumi}.}
\tightlist
\item
  Open the \texttt{example.qmd} file in RStudio.
\item
  Make sure you are working on branch \texttt{testbranch}:
  \texttt{git\ branch}
\item
  Add a new line to the bottom:
  \texttt{This\ is\ a\ change\ I\ made\ on\ the\ testbranch.}
\item
  Save the file.
\end{enumerate}

\subsection{Step 3: Stage and Commit the
Changes}\label{step-3-stage-and-commit-the-changes}

\begin{enumerate}
\def\labelenumi{\arabic{enumi}.}
\tightlist
\item
  Check which files were changed: \texttt{git\ status}
\item
  Stage the file: \texttt{git\ add\ example.qmd}
\item
  Commit the change:
  \texttt{git\ commit\ -m\ "Added\ a\ line\ to\ example\ on\ testbranch"}
\end{enumerate}

Your changes made to the file are now saved locally in your local
branch.

\newpage

\section{Amend the Last Commit}\label{amend-the-last-commit}

\subsection{Step 1: Create a Folder inside the
Project}\label{step-1-create-a-folder-inside-the-project}

\begin{enumerate}
\def\labelenumi{\arabic{enumi}.}
\tightlist
\item
  Inside the RStudio project, create a folder called \texttt{data}.
\item
  Place the data into this folder.
\end{enumerate}

\subsection{Step 2: Amend the Last Commit to Include the
Folder}\label{step-2-amend-the-last-commit-to-include-the-folder}

\begin{enumerate}
\def\labelenumi{\arabic{enumi}.}
\tightlist
\item
  Stage and Commit the Changes:
\end{enumerate}

\begin{Shaded}
\begin{Highlighting}[]
\FunctionTok{git}\NormalTok{ add .}
\FunctionTok{git}\NormalTok{ commit }\AttributeTok{{-}{-}amend}
\end{Highlighting}
\end{Shaded}

\begin{enumerate}
\def\labelenumi{\arabic{enumi}.}
\setcounter{enumi}{1}
\tightlist
\item
  Edit the commit message at the top of the text editor. For example:
  ``Added a line to example on testbranch and created a folder''.
\item
  Save the commit and close the editor.
\item
  Push this amended commit to the remote:
  \texttt{git\ push\ -u\ origin\ testbranch}
\end{enumerate}

You should now see the \texttt{testbranch} and the commit messages
appear on GitHub.

\begin{tcolorbox}[enhanced jigsaw, opacityback=0, coltitle=black, bottomrule=.15mm, opacitybacktitle=0.6, breakable, colbacktitle=quarto-callout-tip-color!10!white, title=\textcolor{quarto-callout-tip-color}{\faLightbulb}\hspace{0.5em}{Tip}, left=2mm, rightrule=.15mm, toprule=.15mm, leftrule=.75mm, bottomtitle=1mm, toptitle=1mm, colback=white, titlerule=0mm, arc=.35mm, colframe=quarto-callout-tip-color-frame]

We need the \texttt{-u} flag on the \texttt{push} command to set the
remote branch as the default tracking branch. After running this, Git
will remember that our local local is connected to the remote (origin).
From now on, we can simply run \texttt{git\ push} without needing to
specify the remote or branch name again.

\end{tcolorbox}

\section{Create a Conflict}\label{create-a-conflict}

\begin{enumerate}
\def\labelenumi{\arabic{enumi}.}
\tightlist
\item
  Switch back to branch \texttt{main}: \texttt{git\ switch\ main}
\item
  Double check which branch you are currently working on:
  \texttt{git\ branch}
\item
  Open the \texttt{example.qmd} file and edit the exact same line in
  \texttt{main} that you previously changed in \texttt{testbranch} to
  trigger a conflict. For example: ``This line was edited on main.''
\item
  Save the file, then stage and commit the change:
\end{enumerate}

\begin{Shaded}
\begin{Highlighting}[]
\FunctionTok{git}\NormalTok{ add .}
\FunctionTok{git}\NormalTok{ commit }\AttributeTok{{-}m}\StringTok{"Added the same line on main to make a conflict"}
\FunctionTok{git}\NormalTok{ push origin main}
\end{Highlighting}
\end{Shaded}

\newpage

\section{Fix the Merge Conflict}\label{fix-the-merge-conflict}

\subsection{\texorpdfstring{Step 1: Merge the \texttt{testbranch} into
\texttt{main}}{Step 1: Merge the testbranch into main}}\label{step-1-merge-the-testbranch-into-main}

Try to merge the changes in \texttt{testbranch} onto \texttt{main}:
\texttt{git\ merge\ testbranch}

You'll see a merge conflict message that Git can't automatically merge
the file.

\begin{verbatim}
<<<<<<< HEAD
This line was edited on main.
=======
This is a change I made on the testbranch.
>>>>>>> testbranch
\end{verbatim}

This means:

\begin{itemize}
\item
  Everything above \texttt{=======} is from \texttt{main}
\item
  Everything after \texttt{=======} is from \texttt{testbranch}
\end{itemize}

\subsection{Step 2: Resolve the
Conflict}\label{step-2-resolve-the-conflict}

\begin{enumerate}
\def\labelenumi{\arabic{enumi}.}
\tightlist
\item
  Edit the file to keep only one version --- or combine them: ``This
  line includes changes from both branches.''
\item
  Make sure you delete all the conflict markers:
  \texttt{\textless{}\textless{}\textless{}\textless{}\textless{}\textless{}\textless{}\ HEAD},
  \texttt{=======} and
  \texttt{\textgreater{}\textgreater{}\textgreater{}\textgreater{}\textgreater{}\textgreater{}\textgreater{}\ testbranch}
\item
  Save the file.
\end{enumerate}

\subsection{Step 3: Finalize the Merge}\label{step-3-finalize-the-merge}

\begin{enumerate}
\def\labelenumi{\arabic{enumi}.}
\tightlist
\item
  Stage the fixed file: \texttt{git\ add\ .}
\item
  Complete the merge with a commit:
  \texttt{git\ commit\ -m\ "Resolved\ merge\ conflict\ in\ example.qmd"}
\item
  Push the updated \texttt{main} branch to GitHub:
  \texttt{git\ push\ origin\ main}
\end{enumerate}

Now you should see that the GitHub commit history on the \texttt{main}
branch displays all commit messages we've made before (see
Figure~\ref{fig-commit-history}).

\begin{figure}

\centering{

\pandocbounded{\includegraphics[keepaspectratio]{Images/Screenshot-step6-commit-history.png}}

}

\caption{\label{fig-commit-history}Commit History View in Git}

\end{figure}%

\newpage

\section{Make an Annotated Tag}\label{make-an-annotated-tag}

Annotated tags stores extra metadata, such as the tagger name, email and
date.

\begin{enumerate}
\def\labelenumi{\arabic{enumi}.}
\tightlist
\item
  Create a new annotated tag identified with v1.0:
  \texttt{git\ tag\ -a\ v1.0}
\item
  Write a message for the tag in the text editor. For example: ``Initial
  release version''
\item
  Save the editor and close it.
\item
  Push the tag to the remote: \texttt{git\ push\ origin\ v1.0}
\end{enumerate}

Now you can view your tags on GitHub.

\begin{tcolorbox}[enhanced jigsaw, opacityback=0, coltitle=black, bottomrule=.15mm, opacitybacktitle=0.6, breakable, colbacktitle=quarto-callout-tip-color!10!white, title=\textcolor{quarto-callout-tip-color}{\faLightbulb}\hspace{0.5em}{Tip}, left=2mm, rightrule=.15mm, toprule=.15mm, leftrule=.75mm, bottomtitle=1mm, toptitle=1mm, colback=white, titlerule=0mm, arc=.35mm, colframe=quarto-callout-tip-color-frame]

By default, \texttt{git\ tag} creates a tag on \texttt{HEAD} (i.e., the
latest commit). To tag an earlier commit, simply add the commit hash at
the end of the command:

\begin{Shaded}
\begin{Highlighting}[]
\FunctionTok{git}\NormalTok{ tag }\OperatorTok{\textless{}}\NormalTok{tag{-}name}\OperatorTok{\textgreater{}} \OperatorTok{\textless{}}\NormalTok{commit{-}hash}\OperatorTok{\textgreater{}}
\end{Highlighting}
\end{Shaded}

\end{tcolorbox}

\newpage

\section{Delete a Branch}\label{delete-a-branch}

\begin{enumerate}
\def\labelenumi{\arabic{enumi}.}
\tightlist
\item
  Switch to branch \texttt{main}: \texttt{git\ switch\ main}. Remember
  that you can use \texttt{git\ branch} to double check which branch you
  are working on.
\item
  Delete the branch \texttt{testbranch} locally:
  \texttt{git\ branch\ -d\ testbranch}
\item
  Delete the branch from the remote:
  \texttt{git\ push\ origin\ -\/-delete\ testbranch}
\end{enumerate}

Now you can see that there is only one branch \texttt{main} in GitHub.

\begin{tcolorbox}[enhanced jigsaw, opacityback=0, coltitle=black, bottomrule=.15mm, opacitybacktitle=0.6, breakable, colbacktitle=quarto-callout-important-color!10!white, title=\textcolor{quarto-callout-important-color}{\faExclamation}\hspace{0.5em}{Important}, left=2mm, rightrule=.15mm, toprule=.15mm, leftrule=.75mm, bottomtitle=1mm, toptitle=1mm, colback=white, titlerule=0mm, arc=.35mm, colframe=quarto-callout-important-color-frame]

You cannot delete a branch if your \texttt{HEAD} is on that branch.

\end{tcolorbox}

\section{Show the Commit Log}\label{show-the-commit-log}

Use \texttt{git\ log\ -\/-oneline} to show the commit log in the
terminal.

The \texttt{-\/-oneline} flag condenses each commit to a single line. By
default, it displays only the commit ID and the first line of the commit
message. Your typical \texttt{git\ log-\/-oneline} output will look
something like this:

\begin{Shaded}
\begin{Highlighting}[]
\NormalTok{867894f (HEAD {-}\textgreater{} main, tag: v1.0, origin/main) Resolved merge conflict in example.qmd}
\NormalTok{7379f16 Added a line on main}
\NormalTok{4099339 Added a line to example and a data folder on testbranch}
\NormalTok{1116428 Initial commit}
\end{Highlighting}
\end{Shaded}

\newpage

\section{Make a Plot and Undo the
Commit}\label{make-a-plot-and-undo-the-commit}

\subsection{Step 1: Make a Plot Using
ggplot2}\label{step-1-make-a-plot-using-ggplot2}

\begin{enumerate}
\def\labelenumi{\arabic{enumi}.}
\tightlist
\item
  Create a new section called ``Make a Plot'' by using the \texttt{\#}
  character.
\item
  Install the ggplot2 package in the Console then load it:
\end{enumerate}

\begin{Shaded}
\begin{Highlighting}[]
\CommentTok{\# load the package}
\FunctionTok{library}\NormalTok{(ggplot2)}

\CommentTok{\# view the dataset}
\NormalTok{msleep}
\end{Highlighting}
\end{Shaded}

\begin{enumerate}
\def\labelenumi{\arabic{enumi}.}
\setcounter{enumi}{2}
\tightlist
\item
  Try to make a plot by using a built-in dataset in ggplot2:
\end{enumerate}

\begin{Shaded}
\begin{Highlighting}[]
\CommentTok{\# plot for total amount of sleep for mammals}
\FunctionTok{ggplot}\NormalTok{(}\AttributeTok{data =}\NormalTok{ msleep, }\AttributeTok{mapping =} \FunctionTok{aes}\NormalTok{(}\AttributeTok{x =}\NormalTok{ sleep\_total)) }\SpecialCharTok{+}
  \FunctionTok{geom\_density}\NormalTok{() }\SpecialCharTok{+}
  \FunctionTok{labs}\NormalTok{(}\AttributeTok{title =} \StringTok{"Distribution of Total Sleep Time in Mammals"}\NormalTok{, }
       \AttributeTok{x =} \StringTok{"Total Amount of Sleep (hours)"}\NormalTok{,}
       \AttributeTok{y =} \StringTok{"Density"}\NormalTok{) }\SpecialCharTok{+}
  \FunctionTok{theme\_classic}\NormalTok{()}
\end{Highlighting}
\end{Shaded}

\begin{figure}[H]

\centering{

\pandocbounded{\includegraphics[keepaspectratio]{guide_files/figure-pdf/fig-sleep-density-1.pdf}}

}

\caption{\label{fig-sleep-density}Distribution of Total Sleep Time in
Mammals}

\end{figure}%

\subsection{Step 2: Stage and Commit the
Changes}\label{step-2-stage-and-commit-the-changes}

\begin{Shaded}
\begin{Highlighting}[]
\FunctionTok{git}\NormalTok{ add .}
\FunctionTok{git}\NormalTok{ commit }\AttributeTok{{-}m}\StringTok{"Add a new section and make a plot"}
\end{Highlighting}
\end{Shaded}

You can use \texttt{git\ log\ -\/-oneline} to view the commit history.
You'll get something like this:

\begin{Shaded}
\begin{Highlighting}[]
\NormalTok{bd86707 (HEAD {-}\textgreater{} main) Add a new section and make a plot}
\NormalTok{867894f (tag: v1.0, origin/main) Resolved merge conflict in example.qmd}
\NormalTok{7379f16 Added a line on main}
\NormalTok{4099339 Added a line to example and a data folder on testbranch}
\NormalTok{1116428 Initial commit}
\end{Highlighting}
\end{Shaded}

\subsection{Step 3: Undo the Commit}\label{step-3-undo-the-commit}

If you've just made a commit and realize you want to undo it, you can
use the following command --- as long as you haven't pushed the commit
yet:

\begin{Shaded}
\begin{Highlighting}[]
\FunctionTok{git}\NormalTok{ reset HEAD\textasciitilde{}1}
\end{Highlighting}
\end{Shaded}

Now try:

\begin{Shaded}
\begin{Highlighting}[]
\FunctionTok{git}\NormalTok{ log }\AttributeTok{{-}{-}oneline}
\end{Highlighting}
\end{Shaded}

You'll see that the latest commit has been removed --- \texttt{HEAD} has
moved one commit backward.

This command undoes the most recent commit but keeps all the changes in
your working directory, just no longer staged. You can confirm this by
running:

\begin{Shaded}
\begin{Highlighting}[]
\FunctionTok{git}\NormalTok{ status}
\end{Highlighting}
\end{Shaded}

From here, you can make any adjustments, stage the correct files, and
commit again with the proper message or content.

\begin{tcolorbox}[enhanced jigsaw, opacityback=0, coltitle=black, bottomrule=.15mm, opacitybacktitle=0.6, breakable, colbacktitle=quarto-callout-tip-color!10!white, title=\textcolor{quarto-callout-tip-color}{\faLightbulb}\hspace{0.5em}{Tip}, left=2mm, rightrule=.15mm, toprule=.15mm, leftrule=.75mm, bottomtitle=1mm, toptitle=1mm, colback=white, titlerule=0mm, arc=.35mm, colframe=quarto-callout-tip-color-frame]

If you prefer to keep all changes staged, use
\texttt{git\ reset\ -\/-soft} instead.

\end{tcolorbox}




\end{document}
